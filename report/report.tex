\documentclass{article}
\usepackage[utf8]{inputenc}
\usepackage[parfill]{parskip}
\usepackage{amsmath}
\usepackage{amsfonts}
\usepackage{amssymb}
\usepackage[makeroom]{cancel}

\newtheorem{goal}{Goal}

\title{Secure Comparator based on Zero-Knowledge Proof}
\author{Raziq Ramli}
\date{December 2021}
\author{Raziq R. Ramli\\ CS Department, Purdue University}

\begin{document}

\maketitle

\section*{1. Problem Description}

Alice and Bob are subcontractors for the same company. 
They each hold a ~4GB file of all the company clients’ 
passwords and they are supposed to use them to develop 
apps. To make sure the apps are consistent they need 
to ensure that the password files they hold are 
identical. But ... they do not trust each-other!

Your goal is to implement a protocol which will allow 
them to check that the files are identical without any 
of the parties revealing to the other party the contents 
of his file.

\section*{2. Solution}

We can reduce the problem to the socialist millionaire problem, where
two millionaires want to check if they have equal wealth without having to disclose
their amount of wealth. 

Given this, we can implement a protocol that uses the Socialist Millionaire
Protocol (SMP) which is a solution to the socialist millionaire problem, to achieve
this project's goal. 
The general outline for the implemented protocol is as follows:


Initial Routine

\begin{itemize}
  \item Alice computes $m_a = $ SHA3-512(Alice's file)
  \item Bob computes $m_b = $ SHA3-512(Bob's file)
  \item Alice and Bob agrees on a cyclic group $G$ of prime order $q$ (2047-bit), 
    and a generator $g_1$
    (this information is public)
\end{itemize}

SMP - Key Exchange 1
\begin{itemize}
  \item Alice generates random numbers $a_2$ and $a_3$ s.t. $q \nmid a_2, a_3$
  \item Bob generates random numbers $b_2$ and $b_3$ s.t. $q \nmid b_2, b_3$
  \item Alice sends $g_1^{a_2}$ and $g_1^{a_3}$
  \item Bob sends $g_1^{b_2}$ and $g_1^{b_3}$
  \item Alice and Bob compute $g_2 = (g_1^{a_2})^{b_2}$ and $g_3 = (g_1^{a_3})^{b_3}$
\end{itemize}

SMP - Key Exchange 2
\begin{itemize}
  \item Alice generates a random number r
  \item Bob generates a random number s
  \item Alice sends $P_a = g_3^r$ and $Q_a = g_1^rg_2^{m_a}$
  \item Bob sends $P_b = g_3^s$ and $Q_b = g_1^sg_2^{m_b}$
\end{itemize}

SMP - Key Exchange 3
\begin{itemize}
  \item Alice sends $R_a = (Q_a/Q_b)^{a_3} = g_1^{(r-s)a_3}\cdot g_2^{(m_a-m_b)a_3}$
  \item Bob sends $R_b = (Q_a/Q_b)^{b_3} = g_1^{(r-s)b_3}\cdot g_2^{(m_a-m_b)b_3}$
  \item Alice and Bob compute $R_{ab} = ((Q_a/Q_b)^{a_3})^{b_3} = 
  g_1^{(r-s)a_3b_3} \cdot g_2^{(m_a-m_b)a_3b_3}$
\end{itemize}
SMP - Key Comparison
\begin{itemize}
  \item Alice's and Bob's files are equal if $R_{ab} == P_a / P_b$
\end{itemize}

$Proof\ of\ correctness:$
\begin{align*}
  R_{ab} &= P_a / P_b\\
  g_1^{(r-s)a_3b_3} \cdot g_2^{(m_a-m_b)a_3b_3} &= g_1^{(r-s)a_3b_3}\\
  \cancel{g_1^{(r-s)a_3b_3}} \cdot g_1^{(m_a-m_b)a_2b_2a_3b_3} &= \cancel{g_1^{(r-s)a_3b_3}}\\
  g_1^{(m_a-m_b)a_2b_2a_3b_3} &= 1\\
  (m_a-m_b)a_2b_2a_3b_3 &= kq && \text{for some $k \epsilon \mathbb{Z}$, since $g_1$ has order $q$}\\
  (m_a-m_b)a_2b_2a_3b_3 &= 0 && \text{since $q$ is prime \&\ } q \nmid (m_a-m_b), a_2, b_2,a_3, b_3\\
  m_a - m_b &= 0 &&\text{since $a_2, b_2, a_3, b_3 \neq 0$}\\
  m_a &= m_b
\end{align*}

$m_a = m_b$ implies SHA3-512(Alice's file) = SHA3-512(Bob's file).

From the collision-resistance property of SHA3, we can conclude that Alice and Bob 
share the same file.

\section*{3. Security Goals}

Without loss of generality, this report will call the adversary Eve, and will only 
refer to Alice when writing from the perspective of honest parties. 

In addition, note that 
the security goals are designed with the assumption that honest parties may only
perform passive attacks to each other.

\begin{goal}
The protocol should prevent Alice from learning the content of Bob's file, even if 
Alice possesses a full copy of the file (e.g., from past successful comparisons). 
\footnote{Alice may learn the content of Bob's file if the protocol indicated
their files are equal.}
\end{goal}

$Approach:$ 

Socialist Millionaire Protocol, as described in Section 2.

$Proof:$

The Socialist Millionaire Protocol requires Bob to send his hashed file, $m_b$, in the 
form of 
$$Q_b = g_1^sg_2^{m_b}$$ 
$$R_{b} = g_1^{(r-s)b_3} \cdot g_2^{(m_a-m_b)b_3}$$ 

From the hardness of the discrete logarithm problem, these two message do not expose 
the value of $m_b$.

However, suppose Alice knows the content of Bob's past file, $m_b'$ (from past successful 
comparison) the hardness of the discrete logarithm problem does not prevent Alice from 
trying to compute $$Q_b' = g_1^sg_2^{m_b'}$$
$$R_{b}' = g_1^{(r-s)b_3} \cdot g_2^{(m_a-m_b')b_3}$$ 

Assuming Alice can compute $Q_b'$ or $R_b'$, if $Q_b' = Q_b$ or $R_b' = R_b$, 
then Alice can be certain that $m_b = m_b'$.

However, computing $Q_b'$ and $R_b'$ requires Alice to know Bob's secret key $s$, which the 
protocol protects using randomness and the hardness of the discrete logarithm problem. So,
$Q_b'$ and $R_b'$ are not computable by Alice

Thus, Alice cannot use the protocol to learn the content of Bob's file, even if she posseses 
a full copy of it.

\begin{goal}
The protocol should prevent Eve from impersonating as Alice undetected, even if Eve 
has access to the messages previously sent by Alice.
\end{goal}

$Approach:$

The protocol incorporates Ed25519 public-key digital signature system to authenticate 
the channel between Alice and Bob. 

In other words, the protocol requires Alice to sign 
the message that she sends with her private key, and the protocol requires Bob to 
verify the $(message, signature)$ pair that he receives with Alice's public key.

To prevent Eve from randomly sending Alice's past $(message, signature)$ pairs 
to Bob undetected, Alice will append the message that she is sending with the 
message that she recently receives from Bob. Bob, in turn, will verify that he 
receives from Alice is appended with his recently sent message.

If any of the verfication fails, then Bob will abort the protocol.

$Proof:$

From the security of Ed25519 signature system, it is infeasible for Eve to 
forge a valid $(message, signature)$ pair from Alice.

From the randomness of the messages in Socialist Millinoaire's Protocol (and therefore 
the randomness of the suffix in Alice's messages) and the size of the message space in 
Socialist Millionaire Protocol, it is infeasible for Eve to find a valid message 
that has been signed by Alice.

\begin{goal}
The protocol should prevent eavesdropper Eve from learning if Alice and Bob have the 
same file or not.
\end{goal}

$Approach:$ 

Socialist Millionaire Protocol, as described in $Section\ 2$

$Proof:$

The Socialist Millionaire Protocol allows any party to learn the equality of 
Alice's and Bob's files if they can compute both
$$P_a / P_b$$ $$R_{ab} = (Q_a/Q_b)^{a_3b_3}$$

However, to compute $R_{ab}$, Eve needs to obtain Alice's secret key $a_3$
or Bob's secret key $b_3$, which the protocol protects using randomness and 
the hardness of the discrete logarithm problem. So, $R_{ab}$ is not computable
by Eve.

Thus, it is not possible for Eve to learn the equality of Alice's and Bob's files 
through this protocol.

\section*{4. Implementation}

\subsection*{4.1 Code Spec}

Language used: 
\begin{itemize}
  \item Python
\end{itemize}

Libraries used: 
\begin{itemize}
  \item socket - for client-server implementation
  \item hashlib - for hashing messages
  \item secrets - for generating large random keys securely
  \item nacl - for signing/verifying messages
\end{itemize}

\subsection*{4.2. Assumptions}

For the protocol implementation, is assumed that:

\begin{itemize}
  \item Alice and Bob knows each others' public keys in advanced
  \item Alice and Bob fixes a cyclic group G of prime number $q$ and generator $g_1$ 
  for the Socialist Millionaire Protocol in advanced
\end{itemize}

\subsection*{4.3. Note}

For simplicity, the implemented server only allows 2 connections at a time. 
This may allow an adversary to monopolize the server connections and deny 
services to Alice or Bob. 

However, since this project is only concerned with 
data secrecy and integrity, this type of attack is not covered in the 
implementation.

\end{document}