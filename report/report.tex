\documentclass{article}
\usepackage[utf8]{inputenc}
\usepackage[parfill]{parskip}
\usepackage{amsmath}
\usepackage{amsfonts}
\usepackage{amssymb}
\usepackage[makeroom]{cancel}

\newtheorem{goal}{Goal}

\title{Secure Comparator based on Zero-Knowledge Proof}
\author{Raziq Ramli}
\date{November 2021}
\author{Raziq R. Ramli\\ CS Department, Purdue University}

\begin{document}

\maketitle

\section*{1. Problem Description}

Alice and Bob are subcontractors for the same company. 
They each hold a ~4GB file of all the company clients’ 
passwords and they are supposed to use them to develop 
apps. To make sure the apps are consistent they need 
to ensure that the password files they hold are 
identical. But ... they do not trust each-other!

Your goal is to implement a protocol which will allow 
them to check that the files are identical without any 
of the parties revealing to the other party the contents 
of his file.

\section*{2. Approach}

I reduced the problem to the Socialist Millionaire Problem, where
two millionaires want to check if they have equal wealth without having to disclose
their amount of wealth. 

Then, I incorporated an existing solution to the problem called Socialist Millionaire
Protocol (SMP) as a routine in my protocol. The general outline for my protocol is as 
follows.


Initial Routine

\begin{itemize}
  \item Alice computes $m_a = $ SHA3-512(Alice's file)
  \item Bob computes $m_b = $ SHA3-512(Bob's file)
  \item Alice and Bob agrees on a cyclic group $G$ of prime order $q$ (2047-bit), 
    and a generator $g_1$
    (this information is public)
\end{itemize}

SMP - Key Exchange 1
\begin{itemize}
  \item Alice generates random numbers $a_2$ and $a_3$ s.t. $q \nmid a_2, a_3$
  \item Bob generates random numbers $b_2$ and $b_3$ s.t. $q \nmid b_2, b_3$
  \item Alice sends $g_1^{a_2}$ and $g_1^{a_3}$
  \item Bob sends $g_1^{b_2}$ and $g_1^{b_3}$
  \item Alice and Bob compute $g_2 = (g_1^{a_2})^{b_2}$ and $g_3 = (g_1^{a_3})^{b_3}$
\end{itemize}

SMP - Key Exchange 2
\begin{itemize}
  \item Alice generates a random number r
  \item Bob generates a random number s
  \item Alice sends $P_a = g_3^r$ and $Q_a = g_1^rg_2^{m_a}$
  \item Bob sends $P_b = g_3^s$ and $Q_b = g_1^sg_2^{m_b}$
\end{itemize}

SMP - Key Exchange 3
\begin{itemize}
  \item Alice sends $R_a = (Q_a/Q_b)^{a_3} = g_1^{(r-s)a_3}\cdot g_2^{(m_a-m_b)a_3}$
  \item Bob sends $R_b = (Q_a/Q_b)^{b_3} = g_1^{(r-s)b_3}\cdot g_2^{(m_a-m_b)b_3}$
  \item Alice and Bob compute $R_{ab} = ((Q_a/Q_b)^{a_3})^{b_3} = 
  g_1^{(r-s)a_3b_3} \cdot g_2^{(m_a-m_b)a_3b_3}$
\end{itemize}
SMP - Key Comparison
\begin{itemize}
  \item Alice's and Bob's files are equal if $R_{ab} == P_a / P_b$
\end{itemize}

$Proof\ of\ correctness:$
\begin{align*}
  R_{ab} &= P_a / P_b\\
  g_1^{(r-s)a_3b_3} \cdot g_2^{(m_a-m_b)a_3b_3} &= g_1^{(r-s)a_3b_3}\\
  \cancel{g_1^{(r-s)a_3b_3}} \cdot g_1^{(m_a-m_b)a_2b_2a_3b_3} &= \cancel{g_1^{(r-s)a_3b_3}}\\
  g_1^{(m_a-m_b)a_2b_2a_3b_3} &= 1\\
  (m_a-m_b)a_2b_2a_3b_3 &= kq && \text{for some $k \epsilon \mathbb{Z}$, since $g_1$ has order $q$}\\
  (m_a-m_b)a_2b_2a_3b_3 &= 0 && \text{since $q$ is prime \&\ } q \nmid (m_a-m_b), a_2, b_2,a_3, b_3\\
  m_a - m_b &= 0 &&\text{since $a_2, b_2, a_3, b_3 \neq 0$}\\
  m_a &= m_b
\end{align*}

$m_a = m_b$ implies SHA3-512(Alice's file) = SHA3-512(Bob's file).

From the collision-resistance property of SHA3, we can conclude with a high probability 
that Alice and Bob share the same file.

\section*{3. Security Goals}

For simplicity, I will call our adversary Eve, and will only refer to Alice
when writing from the perspective of honest parties (without loss of generality).

\begin{goal}
Prevent Alice from learning the content of Bob's file over this protocol, even if 
Alice possesses a full copy of Bob's file (say, from past successful comparisons).
\end{goal}

$Caveat:$ 

Alice may learn the content of Bob's file if the protocol indicated their files 
are equal.

$Approach:$ 

Socialist Millionaire Protocol, as described in Section 2.

$Proof:$

If Bob's file is not equal to Alice's file, we will prove that Alice will not be able 
to learn anything about the file that Bob has sent over this protocol.

The protocol requires Bob to send his file $m_b$, to Alice in the form of 
$$Q_b = g_1^sg_2^{m_b}$$ 
$$R_{b} = g_1^{(r-s)b_3} \cdot g_2^{(m_a-m_b)b_3}$$ 

From the hardness of Discrete Logarithm problem, these two message do not expose the content of 
$m_b$.

However, suppose Alice knows the content of Bob's old file, $m_b'$, the hardness of 
Discrete Logarithm problem does not prevent Alice from trying 
to compute $$Q_b' = g_1^sg_2^{m_b'}$$
$$R_{b}' = g_1^{(r-s)b_3} \cdot g_2^{(m_a-m_b')b_3}$$ 

If the computed $Q_b' = Q_b$ or $R_b' = R_b'$, then Alice can be certain that $m_b = m_b'$.

However, computing $Q_b'$ or $R_b'$ requires Alice to know Bob's secret key $s$, which is 
not exposed by the protocol. Thus, $Q_b'$ and $R_b'$ are not 
computable by Alice, and Alice cannot learn about the file that Bob has sent over 
this protocol is equal to her copy of Bob's password file.

$Practical\ consequence:$

Suppose Alice and Bob interacts using this protocol every quarter. The first quarter,
they learn that their files are equal. The second quarter, Alice's file has changed while
Bob's file remains the same. In this case, Alice technically has a full copy of Bob's file,
but she will not be able to verify if Bob is still using this file over this protocol. 

\begin{goal}
Even if Eve has access to the messages previously sent by Alice, Eve should not
be able to impersonate as Alice to Bob undetected.
\end{goal}

$Approach:$

To prevent Eve from sending forged messages to Bob undetected, Alice will always sign the 
message that she sends with her private key, and Bob will always verify the signature that 
he receives with Alice's public key. The exact public-key digital signature system 
being used is Ed25519.

To prevent Eve from randomly sending Alice's old messages to Bob undetected, Alice will 
always append the message that she's sending with the message that she recently receives 
from Bob. Bob, in turn, will always verify that the message he receives contains 
the message that he recently sent as a suffix.

Supposed Bob detects any forgery or invalid suffix, then he will abort the protocol.

$Proof:$

From the security of Ed25519 signature system, it is infeasible for Eve to 
forge a valid $(message, signature)$ pair.

From the randomness of the messages in Socialist Millinoaire's Protocol (and therefore 
the randomness of the suffix in Alice's messages) and the size of the message space in 
Socialist Millionaire's Protocol, 
it is infeasible for Eve to find a valid message that has been signed by Alice.

\begin{goal}
At the end of the protocol, even if eavesdropper Eve knows the full content of Alice's file,
she should not be able to tell if Alice and Bob have the same files or not.
\end{goal}

$Approach:$ 

Socialist Millionaire Protocol, as described in $Section\ 2$

$Proof:$

Eve may only learn that Alice's and Bob's files are equal if she can compute 
$$P_a / P_b$$ and $$R_{ab} = (Q_a/Q_b)^{a_3b_3}$$

However, to compute $R_ab$, Eve would need to obtain Alice's secret key $a_3$
or Bob's secret key $b_3$, which were never exposed anywhere in the protocol.
So, it is not possible for Eve to learn the equality of Alice's and Bob's files 
through this protocol, even if she has the full content of Alice's file.

$Consequence:$ 

Even if Alice's password file was somehow leaked to Eve, this protocol will not allow 
Eve to test if she also knows the full content of Bob's password file.

\section*{4. Code Spec}

Language used: 
\begin{itemize}
  \item Python
\end{itemize}

Libraries used: 
\begin{itemize}
  \item socket - for client-server implementation
  \item hashlib - for hashing messages
  \item secrets - for generating large random keys securely
  \item nacl - for signing/verifying messages
\end{itemize}

\section*{5. Assumptions}

For the protocol implementation, is assumed that:

\begin{itemize}
  \item Alice and Bob knows each others' public keys in advanced
  \item Alice and Bob fixes a cyclic group G of prime number $q$ and generator $g_1$ 
  for the Socialist Millionaire Protocol in advanced
  \item Alice and Bob may only perform passive attacks based on what is on their view
\end{itemize}

\section*{6. Note}

For simplicity, the implemented server only allows 2 connections at a time. 
This may allow an adversary to monopolize the server connections and deny 
services to Alice or Bob. However, since this project is only concerned with 
data secrecy and integrity, this type of attack is not covered in the 
implementation.

\end{document}